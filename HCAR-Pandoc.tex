\begin{hcarentry}[updated]{Pandoc}
\label{pandoc}
\report{John MacFarlane}%11/09
\status{active development}
\participants{Andrea Rossato, Peter Wang, Paulo Tanimoto}
\makeheader

Pandoc aspires to be the swiss army knife of text markup formats: it
can read markdown and (with some limitations) HTML, LaTeX, and
reStructuredText, and it can write markdown, reStructuredText, HTML,
DocBook XML, OpenDocument XML, ODT, RTF, groff man, MediaWiki markup,
GNU Texinfo, LaTeX, ConTeXt, and S5.  Pandoc's markdown syntax includes
extensions for LaTeX math, tables, definition lists, footnotes, and more.

Since the last report, there has been one release (1.2.1).
\begin{itemize}
  \item Users may notice a significant speedup in reading markdown in
    \verb!--smart! mode; the abbreviations parser has been made much more
    efficient.
  \item Default HTML output now wraps sections in divs with unique
    identifiers.  This should aid manipulation using javascript and
    other tools.
  \item We have made some progress in replacing the old POSIX shell
    script wrappers with more portable Haskell wrappers.
\end{itemize}

\FurtherReading
    \url{http://johnmacfarlane.net/pandoc/}
\end{hcarentry}
