% Pandoc-JP.tex
\begin{hcarentry}{Pandoc}
\label{pandoc}
\report{John MacFarlane}%05/11
\status{active development}
\participants{Andrea Rossato, Peter Wang, Paulo Tanimoto, Eric Kow,
Luke Plant, Justin Bogner, Paul Rivier, Nathan Gass, Puneeth Chaganti,
Josef Svenningsson, Etienne Millon, Joost Kremers}
\makeheader

Pandoc aspires to be the swiss army knife of text markup formats: it
can read markdown and (with some limitations) HTML, LaTeX, Textile, and
reStructuredText, and it can write markdown, reStructuredText, HTML,
DocBook XML, OpenDocument XML, ODT, RTF, groff man, MediaWiki markup,
GNU Texinfo, LaTeX, ConTeXt, EPUB, Textile, Emacs org-mode,
Slidy, and S5. Pandoc's markdown syntax includes extensions for LaTeX math,
tables, definition lists, footnotes, and more.

Since the last report, many new features have been added and improvements
made.  Some highlights:
\begin{compactitem}
\item Support for Textile input and output.
\item Support for Emacs org-mode output.
\item A new ``builder'' module for constructing Pandoc documents programatically.
\item Support for \LaTeX math macros in markdown documents.
\item Support for automatic citations and bibliographies using Andrea
Rossato's citeproc-hs library.
\end{compactitem}

These last two changes bring two of the most powerful features of \LaTeX
to pandoc.

\FurtherReading
    \url{http://pandoc.org}
\end{hcarentry}
