\begin{hcarentry}[updated]{Pandoc}
\label{pandoc}
\report{John MacFarlane}%11/09
\status{active development}
\participants{John MacFarlane, Andrea Rossato, Peter Wang, Paulo Tanimoto, Eric Kow,
Luke Plant, Justin Bogner}
\makeheader

Pandoc aspires to be the swiss army knife of text markup formats: it
can read markdown and (with some limitations) HTML, LaTeX, and
reStructuredText, and it can write markdown, reStructuredText, HTML,
DocBook XML, OpenDocument XML, ODT, RTF, groff man, MediaWiki markup,
GNU Texinfo, LaTeX, ConTeXt, and S5.  Pandoc's markdown syntax includes
extensions for LaTeX math, tables, definition lists, footnotes, and more.

There have been several releases since the last report, with many
bug fixes and small improvements. There are two big architectural
changes. First, pandoc no longer requires Template Haskell, which should
make it more portable. Second, a new, flexible template system has been
added, allowing users much more control over document headers and footers.
Other major changes include support for xetex, support for reST tables,
support for tables without header rows, support for formatting
math as MathML, a new ``plain text'' output format, and a much
more permissive HTML parser. The old \verb!hsmarkdown! and
\verb!html2markdown! scripts have been removed; \verb!pandoc! itself can
now do the work of \verb!html2markdown!. Summaries of the new features
in each release are available on the (newly redesigned) website, along
with full documentation and a new tutorial on using the pandoc
library for structured text manipulation.

\FurtherReading
    \url{http://johnmacfarlane.net/pandoc/}
\end{hcarentry}
