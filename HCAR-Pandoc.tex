\begin{hcarentry}[updated]{Pandoc}
\label{pandoc}
\report{John MacFarlane}%05/09
\status{active development}
\participants{Recai Okta\c{s}, Andrea Rossato, Peter Wang}
\makeheader

Pandoc aspires to be the swiss army knife of text markup formats: it
can read markdown and (with some limitations) HTML, LaTeX, and
reStructuredText, and it can write markdown, reStructuredText, HTML,
DocBook XML, OpenDocument XML, ODT, RTF, groff man, MediaWiki markup,
GNU Texinfo, LaTeX, ConTeXt, and S5.  Pandoc's markdown syntax includes
extensions for LaTeX math, tables, definition lists, footnotes, and more.

Since the last report, there have been two releases of pandoc (1.1 and 1.2),
including many bug fixes and the following new features:
\begin{itemize}
  \item Support for literate Haskell.
  \item New \texttt{--jsmath} and \texttt{--email-obfuscation} options.
  \item Better CSS styling in HTML tables.
  \item Windows installer no longer requires admin privileges.
  \item Support for citeproc-hs-0.2.
\end{itemize}

\FurtherReading
    \url{http://johnmacfarlane.net/pandoc/}
\end{hcarentry}
